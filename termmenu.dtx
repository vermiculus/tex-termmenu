% \iffalse meta-comment
% arara: clean: { files: [ termmenu.aux, termmenu.glo, termmenu.hd, termmenu.idx, termmenu.ilg, termmenu.ind, termmenu.out ] }
% arara: pdflatex
% !arara: makeindex
% !arara: pdflatex
% arara: pdflatex
%
%%
%% File: termmenu.dtx Copyright(C) 2015 Sean Allred
%%
%% termmenu.dtx may be distributed and/or modified under the
%% conditions of the LaTeX Project Public License (LPPL),
%% either version 1.3c of this license or (at your option)
%% any later version.  The latest version of this license is
%% in the file
%%
%%    http://www.latex-project.org/lppl.txt
%%
%% The released version is available from CTAN.
%%
%% -----------------------------------------------------------------------
%%
%% The development version can be found at
%%
%%    http://www.github.com/vermiculus/tex-termmenu
%%
%% for those people who are interested.
%%
%% -----------------------------------------------------------------------
%%
%
%<*driver>
\documentclass[full]{l3doc}
%</driver>
%<*driver|package>
\def\ExplFileName{termmenu}
\def\ExplFileVersion{0.2}
\def\ExplFileDate{2015-05-25}
\def\ExplFileDescription{Terminal-driven menu support}
\def\ExplFileExtension{dtx}
%</driver|package>
%<*driver>
\begin{document}
  \DocInput{\jobname.dtx}
\end{document}
%</driver>
% \fi
%
% \title{^^A
%   The \textsf{termmenu} package\\ Terminal-driven menu support^^A
%   \thanks{This file describes v\ExplFileVersion,
%     last revised \ExplFileDate.}^^A
% }
%
% \author{^^A
%  Sean Allred\thanks
%    {^^A
%      E-mail:
%        \href{mailto:tex@seanallred.com}
%          {tex@seanallred.com}^^A
%    }^^A
% }
%
% \date{Released \ExplFileDate}
%
% \maketitle
%
% \begin{documentation}
%
%   This module provides simple support for terminal-driven menus in
%   \pkg{expl3}.
%
%   \paragraph{Example of use}
%    \begin{verbatim}
% \termmenu_new:N \g_demo_termmenu
% \termmenu_set_name:Nn \g_demo_termmenu { Demo }
%
% \termmenu_add:Nnnn \g_demo_termmenu { d, duck }
%   { Oh,~you~know...~:) }
%   { \msg_term:n { Quack! } }
%
% \termmenu_do:N \g_demo_termmenu
%
% \bye
%    \end{verbatim}
%    \begin{verbatim}
% $ pdftex demo.tex
% *************************************************
% * Demo
% *************************************************
%
% The following commands are available:
%
%     > d, duck
%         Oh, you know... :)
%
% \choice=d
% *************************************************
% * Quack!
% *************************************************
%    \end{verbatim}
%
% \newpage
%
% Menus are effectively documented property lists with a fancy,
% user-friendly version of \cs{prop_show:N}.
%
% \section{Creating and initializing menus}
%
% \begin{function}[added = 2015-05-23]{\termmenu_new:N}
%   \begin{syntax}
%     \cs{termmenu_new:N} \meta{menu}
%   \end{syntax}
%   Creates a new \meta{menu} or raises an error if the name is
%   already taken.  The declaration is global.  Initially, the menu
%   will be empty.
% \end{function}
%
% \begin{function}[added = 2015-05-23]{\termmenu_set_name:Nn}
%   \begin{syntax}
%     \cs{termmenu_set_name:Nn} \meta{menu} \Arg{name}
%   \end{syntax}
%   Give \meta{menu} a human-friendly name.  When a menu is being
%   presented, \Arg{name} will appear as a title.  If no name is
%   present, no name will be used.
% \end{function}
%
% \begin{function}[added = 2015-05-23]{\termmenu_add:Nnnn}
%   \begin{syntax}
%     \cs{termmenu_add:Nnnn} \meta{menu} \Arg{option} \Arg{help text} \Arg{action}
%   \end{syntax}
%   Insert \meta{option} into a \meta{menu} and provide \meta{help
%   text}.  When a menu is being presented, both \meta{option} and
%   \meta{help text} will be shown.  If \meta{option} is acted upon
%   (i.e. with \cs{termmenu_exec:Nn}), \meta{action} will be inserted
%   into the input stream.
%
%   To simplify the user experience, \meta{option} can be a
%   comma-separated list of synonymous options.  This is often used to
%   create \enquote*{shortcuts} to functionality; instead of always
%   typing out |option|, the end-user can simply say |o| if something
%   like the following is used:
%   \begin{verbatim}
% \termmenu_add:Nnnn \g_tmpa_termmenu { o, option } { ... } { ... }
%   \end{verbatim}
% \end{function}
%
% \section{Using menus}
%
% To allow for ad-hoc processing based on a user's choices, menus are
% shown and acted upon in four separate phases: display, input,
% retrieval, insertion.
%
% \begin{enumerate}
% \item The display phase displays the menu to the end-user.
% \item Then, an input routine is called upon to store information
%   into a specific variable while prompting for a different one.
%   (This is useful to keep the interface clean.  The internal
%   variable can be very strange-looking, but the user can see
%   whatever you which them to see.)
% \item The retrieval phase receives the user's input and looks it up
%   in the list of option synonyms, returning the matching menu entry.
% \item When the entry has been found, the appropriate value is either
%   returned to the programmer or inserted into the input stream.
% \end{enumerate}
% 
% \begin{function}[added = 2015-05-24]{\termmenu_do:NNNNN,
%                                      \termmenu_do:NNNN,
%                                      \termmenu_do:NNN,
%                                      \termmenu_do:NN,
%                                      \termmenu_do:N}
%   \begin{syntax}
%     \cs{termmenu_do:NNNNN} \meta{menu} \meta{prompt token} \meta{input tl} \meta{entry tl} \meta{value tl}
%     \cs{termmenu_do:NNNN}  \meta{menu} \meta{prompt token} \meta{input tl} \meta{entry tl}
%     \cs{termmenu_do:NNN}   \meta{menu} \meta{input tl} \meta{value tl}
%     \cs{termmenu_do:NN}    \meta{menu} \meta{prompt token}
%     \cs{termmenu_do:N}     \meta{menu}
%   \end{syntax}
%   The \cs{termmenu_do:} family of functions are convenience wrappers
%   around the basic display, prompt, and lookup functions.
%   
%   |do:NNNNN| uses \meta{menu} to prompt the user for \meta{prompt
%   token}, setting their input to \meta{input tl}, setting the
%   matched entry to \meta{entry tl}, and storing the associated value
%   in \meta{value tl}.
%
%   |do:NNNN| is the same as |do:NNNNN|, but instead of storing the
%   output in a token list, it is inserted into the input stread.
%
%   |do:NNN| is similar to |do:NNNNN|, but does not allow the
%   programmer to set what the user sees as a prompt.  Instead, the
%   generic \cs{choice} name is used for \meta{input tl}.
%   Additionally, the associated menu entry is not stored.
%
%   |do:NN| is like |do:NNNN|, but does not store either the user's
%   input or its associated menu entry.  The only parameter this takes
%   is the \meta{prompt token}.
%
%   |do:N| is the simplest way to use a menu.  It uses \cs{choice} for
%   the \meta{prompt token} and places the associated value in the
%   input stream.
% \end{function}
%
% \begin{function}[added = 2015-05-24]{\termmenu_get:NNN}
%   \begin{syntax}
%     \cs{termmenu_get:NNN} \meta{menu} \meta{prompt} \meta{tl var}
%   \end{syntax}
%   Present \meta{menu} to the user and prompt for a choice, using
%   \meta{prompt} as the token list read.  \meta{tl var} is set
%   appropriately by \cs{termmenu_find_entry:NnN}.
% \end{function}
%
% \begin{function}[added = 2015-05-23, updated = 2015-05-24]{\termmenu_get_choice:NN}
%   \begin{syntax}
%     \cs{termmenu_get_choice:NN} \meta{menu} \meta{tl var}
%   \end{syntax}
%   This is a wrapper for \cs{termmenu_get:NNN}, except the
%   token \cs{choice} is used as a prompt-value.  \cs{choice} is not
%   overwritten by this function.
% \end{function}
%
% \begin{function}[added = 2015-05-23]{\termmenu_exec:Nn, \termmenu_exec:NV}
%   \begin{syntax}
%     \cs{termmenu_exec:Nn} \meta{menu} \Arg{choice}
%   \end{syntax}
%   Executes the associated action for \meta{choice}.  See
%   \cs{termmenu_add:Nnnn}.
% \end{function}
%
% \begin{function}[added = 2015-05-24]{\termmenu_find_entry:NnN}
%   \begin{syntax}
%     \cs{termmenu_get:NNN} \meta{menu} \Arg{query} \meta{tl var}
%   \end{syntax}
%   If \meta{query} matched an entry in \meta{menu}, this entry is
%   locally placed in \meta{tl var}.  Otherwise, \meta{tl var} will be
%   \cs{q_no_value}.
% \end{function}
%
% \section{Customizing output}
%
% \pkg{termmenu} has limited support for customization.
%
% \begin{variable}[added = 2015-05-23]{\l_termmenu_prompt_tl}
%   This is the message displayed to the user when a menu is
%   presented.  The default value is
%   |The~following~commands~are~available:|.
% \end{variable}
%
% \section{Inspection}
%
% \begin{function}[added = 2015-05-23]{\termmenu_show:N}
%   \begin{syntax}
%     \cs{termmenu_show:N} \meta{arg}
%   \end{syntax}
%   Show the contents of the menu.  Right now, this is the same as
%   \cs{prop_show:N}.
% \end{function}
%
% \section{Internal variables and functions}
%
% \begin{variable}[added = 2015-05-23]{\l__termmenu_spec_tl}
%   This scratch variable holds the documentation and
%   argument for an option when retrieved from the property
%   list.
% \end{variable}
%
% \begin{variable}[added = 2015-05-23]{\g__termmenu_doc_tl}
%   This scratch variable stores the documentation for an option.
%   It is necessary because none of |f|, |x|, |o|-type expansions
%   seem to work where they need to.
%^^A For reference, this place is \cs{@@_write_out:N}.
% \end{variable}
%
% \begin{variable}[added = 2015-05-23]{\g__termmenu_names_prop}
%   This property list stores the names of each menu.  The keys of the
%   property lists are menus (e.g., \cs{g_demo_termmenu}) and the
%   values are their names.
% \end{variable}
%
% \begin{variable}{\g__termmenu_tmp_tl}
%   This scratch variable is used to help set \meta{tl var} from
%   \cs{termmenu_find:NnN}.
% \end{variable}
%
% \begin{variable}[added = 2015-05-24]{\g__termmenu_opt_bool}
%   This scratch variable is used to signal if the user's input was
%   able to match against a known option.
% \end{variable}
%
% \begin{function}[added = 2015-05-24]{\termmenu_display:N}
%   \begin{syntax}
%     \cs{termmenu_display:N} \meta{menu}
%   \end{syntax}
%   Writes out \meta{menu} to the terminal.  No associated actions
%   are performed.  If \meta{menu} has a name (i.e., an entry in
%   \cs{__termmenu_names_prop}), print it as well.
% \end{function}
%
% \end{documentation}
%
% \begin{implementation}
%
% \section{Implementation}
%
% Before I begin, I want to establish a few definitions:
% \begin{description}
% \item[menu] The over-arching data structure that holds (nearly) all
%   related information for a given menu.
% \item[entry] A full, comma-separated specification of valid
%   inputs for a given menu item.
% \item[option] A valid input for an entry.
% \item[value] The associated action for an entry.
% \item[prompt] The macro read in by \TeX{} on a low level; this
%   macro's name is presented to the user in output.
% \end{description}
%
%    \begin{macrocode}
%<*package>
%    \end{macrocode}
%    \begin{macrocode}
%<@@=termmenu>
%    \end{macrocode}
%  
% \begin{macro}{\termmenu_new:N, \termmenu_show:N, \termmenu_add:Nnnn}
%   Each menu is implemented as a property list of options mapped
%   to documentation and actions.  Each option is a property list
%   key.  Since the there is no good way to distinguish between
%   when documentation ends and an associated action begins, the
%   documentation is placed in a group at the head of the key's
%   value.\footnote{This could also be done with sequences, but I
%   consider that unnecessary overhead and complication.}
%    \begin{macrocode}
\cs_set_eq:NN \termmenu_new:N \prop_new:N
\cs_set_eq:NN \termmenu_show:N \prop_show:N
\cs_new_nopar:Nn \termmenu_add:Nnnn
  { \prop_put:Nnn #1 {#2} { {#3} #4 } }
%    \end{macrocode}
% \end{macro}
%
% \begin{variable}{\l_@@_spec_tl}
%   This scratch variable holds the documentation and
%   argument for an option when retrieved from the property
%   list.  It is also used to hold the title of a menu.
%    \begin{macrocode}
\tl_new:N \l_@@_spec_tl
%    \end{macrocode}
% \end{variable}
%
% \begin{macro}{\termmenu_do:NNNNN,
%               \termmenu_do:NNNN,
%               \termmenu_do:NNN,
%               \termmenu_do:NN,
%               \termmenu_do:N}
%   ^^A NNNNN <in menu> <out input> <in prompt> <out entry> <out value>
%   ^^A NNNN  <in menu> <out input> <in prompt> <out entry> >>value
%   ^^A NNN   <in menu> <out input> <out value>
%   ^^A NN    <in menu> <in prompt>
%   ^^A N     <in menu>
%    \begin{macrocode}
\tl_new:N \l_@@_tmp_tl
\tl_new:N \l_@@_value_tl
\cs_new_nopar:Nn \termmenu_do:NNNNN
  {
    \termmenu_display:N #1
    \termmenu_prompt:NN #2 #3
    \termmenu_entry:NVN #1 #2 #4
    \termmenu_value:NVN #1 #4 #5
  }
\cs_new_nopar:Nn \termmenu_do:NNNN
  {
    \tl_clear:N \l_@@_value_tl
    \termmenu_do:NNNNN #1 #2 #3 #4 \l_@@_value_tl
    \tl_use:N \l_@@_value_tl
  }
\cs_new_nopar:Nn \termmenu_do:NNN
  { \termmenu_do:NNNNN #1 #2 \choice \l_@@_tmp_tl #3 }
\cs_new_nopar:Nn \termmenu_do:NN
  { \termmenu_do:NNNN #1 \l_@@_tmp_tl #2 \l_@@_tmp_tl }
\cs_new_nopar:Nn \termmenu_do:N
  { \termmenu_do:NN #1 \choice }
%    \end{macrocode}
% \end{macro}
%
% \begin{macro}{\termmenu_value:NnN}
%   This macro retrieves the action for |#2| in |#1| and places it in
%   |#3|.  Remember that 
%    \begin{macrocode}
\cs_new_nopar:Nn \termmenu_value:NnN
  {
    \prop_get:NnN #1 {#2} \l_@@_spec_tl
    \tl_set:Nf #3 { \tl_tail:N \l_@@_spec_tl } %@todo test expansion
  }
\cs_new_nopar:Nn \termmenu_value:Nn
  {
    \termmenu_value:NnN #1 {#2} \l_@@_tmp_tl
    \tl_use:N \l_@@_tmp_tl
  }
\cs_generate_variant:Nn \termmenu_value:NnN { NVN }
%    \end{macrocode}
%  \end{macro}
%  
%  \begin{macro}{\termmenu_doc:NnN, \termmenu_doc_:Nn}
%    These functions retrieve the help string for entry |#2| in |#1|.
%    \begin{macrocode}
\cs_new_nopar:Nn \termmenu_doc:NnN
  {
    \prop_get:NnN #1 {#2} \l_@@_spec_tl
    \tl_set:Nf #3 { \tl_head:N \l_@@_spec_tl } %@todo test expansion
  }
% ^^A\cs_new_nopar:Nn \termmenu_doc:Nn
% ^^A  {
% ^^A    \msg_term:n { #1 ~,~ #2 }
% ^^A    \termmenu_doc:NnN #1 {#2} \l_@@_tmp_tl
% ^^A    \tl_use:N \l_@@_tmp_tl
% ^^A  }
%    \end{macrocode}
% \end{macro}
%
% \begin{variable}{\g_@@_tmp_tl}
%   This scratch variable is used to escape the groups of mapping
%   constructs (like \cs{prop_map_inline:Nn}).  See
%   \cs{termmenu_find_entry:NnN}.
%    \begin{macrocode}
\tl_new:N \g_@@_tmp_tl
%    \end{macrocode}
% \end{variable}
%
% \begin{variable}{\g_@@_opt_bool}
%   This scratch variable is used to signal if a suitable option was
%   found when searching for a match based on user input.  Since it is
%   used inside two inlined mappings, it is most straightforward for
%   all assignments to be global.
%    \begin{macrocode}
\bool_new:N \g_@@_opt_bool
%    \end{macrocode}
% \end{variable}
%
% \begin{macro}{\termmenu_entry:NnN}
%   Using the menu in |#1|, find a match for |#2| and stick that match
%   in |#3|.
%    \begin{macrocode}
\cs_new_nopar:Nn \termmenu_entry:NnN
  {
    \prop_map_inline:Nn #1
      {
%    \end{macrocode}
% If the input value is in the list of synonyms, set our output value
% to the full list, indicate that we've found a match, and break out
% of the loop.  Note that we use global variables to free ourselves
% from the confines of the group.
%    \begin{macrocode}
        \clist_if_in:nnT {##1} {#2}
          {
            \tl_gset:Nn \g_@@_tmp_tl {##1}
            \bool_gset_true:N \g_@@_opt_bool
            \prop_map_break:
          }
      }
%    \end{macrocode}
% If we haven't set \cs{g_@@_opt_bool} by the time we've finished
% looking, we never found anything.  Set |#3| to \cs{q_no_value} to
% indicate the lack of a match and reset the value of
% \cs{g_@@_opt_bool}.
%    \begin{macrocode}
    \bool_if:NTF \g_@@_opt_bool
      { \tl_set_eq:NN #3 \g_@@_tmp_tl }
      { \tl_set:Nn #3 { \q_no_value } }
    \bool_gset_false:N \g_@@_opt_bool
  }
\cs_generate_variant:Nn \termmenu_entry:NnN { NVN }
%    \end{macrocode}
% \end{macro}
%
% \begin{macro}{\termmenu_prompt_nice:NNN}
%   This function is a little interesting.  In order to keep the
%   end-user's interface clean, we can't simply prompt for the
%   destination variable.  Say, if the destination variable were
%   something like \cs{l__some_confusing_variable_name}, this would
%   cause \TeX{} to display that confusing variable name to the
%   end-user as part of the prompt.  Instead, we ask for |#2|
%   (starting a new group to avoid clobbering any existing definition)
%   and store the result in \cs{out}.  Now we have to get this value
%   outside the group and into |#3|.  The |\expandafter\endgroup|
%   trick works nicely here (and is the
%   official\footnote{\url{http://tex.stackexchange.com/a/246542}} way
%   to do this).
%
% <variable> <prompt>
%    \begin{macrocode}
\cs_new_nopar:Nn \termmenu_prompt:NN
  {
    \group_begin:
      \termmenu_prompt:N #2
    \exp_args:NNNV \group_end:
    \tl_set:Nn #1 #2
  }
% <variable>
\cs_new_nopar:Nn \termmenu_prompt:N
  { \ior_get_str:NN \c_term_ior #1 }
%    \end{macrocode}
% \end{macro}
%
% \begin{macro}{\termmenu_exec:Nn, \termmenu_exec:NV}
%   This function is very simple: it retrieves the value for
%   the option (the key to a property list) and puts it in a
%   scratch variable.  The tail of this scratch variable is
%   then inserted into the input stream.
%
%    \begin{macrocode}
% ^^A \cs_new_nopar:Nn \termmenu_exec:Nn %@kill
% ^^A   {
% ^^A     \termmenu_get_action:NnN #1 {#2} \l_@@_action_tl
% ^^A     \tl_use:N \l_@@_action_tl
% ^^A   }
% ^^A \cs_generate_variant:Nn \termmenu_exec:Nn { NV }
%    \end{macrocode}
% \end{macro}
%
% \begin{variable}{\l_termmenu_prompt_tl}
%   This public variable contains the text to be used as a
%   prompt when displaying a menu.
%    \begin{macrocode}
\tl_new:N \l_termmenu_prompt_tl
\tl_set:Nn \l_termmenu_prompt_tl
 { The~following~commands~are~available: }
%    \end{macrocode}
% \end{variable}
%
% \begin{variable}{\g_@@_names_prop}
%   Globally define the property list to store menu names.
%    \begin{macrocode}
\prop_new:N \g_@@_names_prop
%    \end{macrocode}
% \end{variable}
%
% \begin{macro}{\termmenu_get_name:Nn}
%    \begin{macrocode}
\cs_new_nopar:Nn \termmenu_get_name:NN
 { \prop_get:NnN \g__termmenu_names_prop {#1} #2 }
%    \end{macrocode}
% \end{macro}
%
% \begin{macro}{\termmenu_set_name:Nn}
%   This function names a menu.  An entry is placed in
%   \cs{g_@@_names_prop} with the menu as a key and its name as the
%   value.
%    \begin{macrocode}
\cs_new_nopar:Nn \termmenu_set_name:Nn
 { \prop_put:Nnn \g__termmenu_names_prop {#1} {#2} }
%    \end{macrocode}
% \end{macro}
%
% \begin{variable}{\g_@@_doc_tl}
%   This variable stores the help text for an option.
%    \begin{macrocode}
\tl_new:N \l__termmenu_doc_tl
%    \end{macrocode}
% \end{variable}
%
% \begin{macro}{\termmenu_display:N}
%   Perhaps the only truly complicated part of this package
%   is this function.
%    \begin{macrocode}
\cs_generate_variant:Nn \iow_term:n { V }
\cs_generate_variant:Nn \msg_term:n { V }
\cs_new_nopar:Nn \termmenu_display:N
  {
%    \end{macrocode}
% First, we retrieve the name of the menu.  If it does not
% exist, print a generic \enquote{Menu} header.  If it does
% exist, use \meta{title} as the header.
%    \begin{macrocode}
    \termmenu_get_name:NN #1 \l_@@_spec_tl
    \quark_if_no_value:NTF \l_@@_spec_tl
      { \msg_term:n { Menu } }
      { \msg_term:V \l_@@_spec_tl }
%    \end{macrocode}
% Then, display the prompt.  Note that \cs{iow_term:n}
% without an argument will simply output one blank line.
%    \begin{macrocode}
    \iow_term:n { }
    \iow_term:V \l_termmenu_prompt_tl
    \iow_term:n { }
%    \end{macrocode}
% We want to display each option with its help text in a way that is
% easy to read.
%    \begin{macrocode}
    \prop_map_inline:Nn #1
      {
%    \end{macrocode}
% Now, send wrapped output to the terminal that contains the
% option name indented by four spaces, a new line, and the
% documentation, all wrapped with a running indent of eight
% spaces.  Note that since we introduce a new line with the
% |\\| macro in \cs{iow_wrap:nnnN}, we get the first
% indentation of the documentation for free.
%    \begin{macrocode}
        \clist_set:Nn \l_@@_tmp_clist {##1}
        \termmenu_doc:NnN #1 {##1} \l_@@_tmp_tl
        \iow_wrap:nnnN
          {
            \prg_replicate:nn {4} { \iow_char:N \  }
            > ~ \clist_use:Nn \l_@@_tmp_clist { ,~ } \\
            \tl_use:N \l_@@_tmp_tl
          } { \prg_replicate:nn {8} { \  } } { } \iow_term:n
      }
  }
%    \end{macrocode}
% \end{macro}
%
%    \begin{macrocode}
%</package>
%    \end{macrocode}
%
% \end{implementation}
%
% \PrintIndex
